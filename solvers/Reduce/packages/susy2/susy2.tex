\documentstyle[12pt]{article}
\def\be{\begin{equation}}
\def\ee{\end{equation}}
\def\pe{\begin{eqnarray}}
\def\ke{\end{eqnarray}}
\topmargin=-2cm\textheight=23.5cm\textwidth=16cm
\oddsidemargin=0.25cm
\evensidemargin=0.25cm

\begin{document}

                \title{S U S Y 2}

                \author{by \\
        Ziemowit Popowicz by \\ \\
        Institute of Theoretical Physics, University of Wroc{\l}aw,\\
        pl.M.Borna 9 50-205 Wroc{\l}aw, Poland \\
		e-mail ziemek@ift.uni.wroc.pl \\
                version 1.2}

\maketitle
\begin{abstract}
This package deal, with supersymmetric functions and with algebra
of supersymmetric operators in the extended N=2 as well as in the
nonextended N=1 supersymmery. It allows us
to make  realization of SuSy algebra of differential operators,
compute the gradients of given SuSy Hamiltonians and to obtain
SuSy version of soliton equations using SuSy Lax approach. There
are also many additional procedures also encountered in SuSy soliton
approach, as for example: conjugation of a given SuSy operator, computation
of general form of SuSy Hamiltonians (up to SuSy-divergence equivalence),
checking of the validity of the Jacobi identity for some SuSy
Hamiltonian operators.
\end{abstract}
\section{Introduction}

  The main idea of the supersymmetry (SuSy) is to treat boson and fermion
operators equally [1,2]. This has been realised by introducing the so called
supermultiplets constructed from the boson and fermion operators and
additionally from the Mayorana spinors. Such supermultiplets posses  the
proper transfomations property under the transformation of the Lorentz group.
At the moment we have no experimental confirmations that the supersymmetry
appeare in the nature.

        The idea of using supersymmetry (SuSy) for the
generalization of the soliton equations [3-7] appeared almost in parallel
to the usage of SuSy in the quantum field theory. The first results,
concerning the construction of classical field theories  with fermionic
and bosonic fields
depending on time and one space variable, can be found in [8-12].
In many cases, the  addition of fermions fields does not guarantee that the
final theory becomes SuSy java.invariant and therefore this method was named as
the fermionic extension in order to distinguish it from the fully SuSy method.

        In order to get a SuSy theory  we have to add to a system of k bosonic
equations kN fermion and k(N-1) boson fields (k=1,2,... N=1,2,..) in such a way
that the final theory becomes SuSy java.invariant.  From the soliton point of view
we can distinguish two important classes of the supersymmetric equations:
the non-extended $(N = 1)$ and extended $( N > 1 )$ cases. Consideration of the
extended case may imply new bosonic equations whose properties need further
investigation. This may be viewed as a bonus, but this extended case is no
more fundamental than the non-extended one. The problem of the
supersymmetrization of the nonlinear partial differential equations has its
own history, and at the moment we have no unique solution [13-40].
We can distinguish three different methods of
supersymmetrization, as for example the algebraic, geometric and direct method.

        In the first two cases we are looking for the symmetry group of the
given equation and then we replace this group by the corresponding
SuSy group. As a final product we are able to obtain
SuSy generalization
of the given equation. The classification into  the algebraical or
geometrical approach is connected with the kind of symmetry which appears
in the classical case. For example, if our classical equation could be
described  in terms of the geometrical object then the simple exchange
of the classical symmetry group of this object with its SuSy partner
justifies the name geometric. In the case of algebraic we are looking for the
symmetry group of the equation without any reference to its
geometrical origin. This strategy could be applied to the so called
hidden symmetry as for example in the case of the Toda lattice .
These methods each have advantages and disadvantages. For example,
sometimes we obtain the fermionic extensions. In the case of the
extended supersymmetric Korteweg-de-Vries equation  we have three different
fully SuSy extensions; however only one of them fits to these two
classifications.

In the direct approach we simply replace all
objects which are appear in the evolution equation  by all possible
combinations of the supermultiplets and its superderivative in such a way
that to conserve the conformal dimensions. This is non unique
and we yields many different possibilities. However the
arbitrariness is reduced if we additionally investigate
super-bi-hamiltonian structure or try to find its supersymmetric Lax pair.
In many cases this approach is successful.


The utilization  of the above methods can be helped by
symbolic computer algebraic and for this reason
we prapared the package SuSy2 in the symbolic language
REDUCE [41].

       We have implemented and ordered the superfunctions in our program,
extensively using the concept of `` noncom operator '' in order to implement
the supersymmetric integro - differential operators. The program is meant
to perform the symbolic calculations using either fully supersymmetric
supermultiplets or the components version of our supersymmetry.
We have constructed 25 different commands to allow us to compute
almost all objects encountered in the supersymmetrization procedure
of the soliton equation.

\section{Supersymmetry}

        The basic object in the supersymmetric analysis is the superfield
and the supersymmetric derivative. The superfields are the superfermions or
the superbosons [1]. These fields, in the case of extended N=2 sypersymmetry,
depends, in addition to $ x $  and  $ t $, upon two
anticommuting variables, $\theta_{1}$ and $\theta_{2}$ {~}{~}
($\theta_{2}\theta_{1} = - \theta_{1}\theta_{2} , \theta_{1}^{2}=
\theta_{2}^{2}=0 $ ).
Their Taylor expansion with respect to the $ \theta^{'}s $ is
\be
       b(x,t,\theta_{1},\theta_{2}):=w+\theta_{1}\zeta_{1}+
        \theta_{2}\zeta_{2}+\theta_{2}\theta_{1}u\\,
\ee
in the case of superbosons, while for the superfermions reads
\be
        f(x,t,\theta_{1},\theta_{2}):=\zeta_{1}+\theta_{1}w+
        \theta_{2}u+\theta_{2}\theta_{1}\zeta_{2},
\ee
where $w$ and $u$ are classical (commuting) functions depending  on $ x $ and $ t $ ,
$ \zeta_{1} $ and $ \zeta_{2} $ are odd Grassmann valued functions  depending
on $ x $ and $ t $.

        In the set of these superfunctions we can defined the usual derivative
and the superderivative. Usually, we encounter two different realizations
of the superderivative : the first we call  `` traditional '' and the second
`` chiral ''.

The traditional realization can be defined by introducing
two superderivatives $ D_{1} $ and $ D_{2} $
\pe
        D_{1} &=& \partial_{\theta_{1}}+\theta_{1}\partial,\\
        D_{2} &=& \partial_{\theta_{2}}+\theta_{2}\partial,
\ke
with the properties:
\pe
        D_{1}*D_{1}=D_{2}*D_{2} = \partial ,  \\
        D_{1}*D_{2} + D_{2}*D_{1} = 0 .
\ke
        The chiral denoted is by
\pe
        D_{1} &=& \partial_{\theta_{1}} - \frac{1}{2}\theta_{2}\partial,\\
        D_{2} &=& \partial_{\theta_{2}} - \frac{1}{2}\theta_{1}\partial,
\ke
with the properties:
\pe
        D_{1}*D_{1}=D_{2}*D_{2} = 0 ,  \\
        D_{1}*D_{2} + D_{2}*D_{1} = -\partial .
\ke

Below we shall use the name `` traditional'' or `` chiral '' or
`` chiral1 ''algebras to denote  kind of the commutation realations on the
superderivativeis assumed.  The `` chiral1 '' algebras case possess,
additioanly to the ``chiral '' algebra, the commutator of $D_{1}$ and
$D_{2}$ denoted  as
\pe
        D_{3} = D_{1}*D_{2} -D_{2}*D_{1}.
\ke

In SuSy2 package we have will implemented the superfunctions and the
algebra of superderivatives. Moreover, we have defined many additional
procedures  which are useful in the supersymetrizations of the classical
nonlinear system of partial differential equation. Different applications
of this package to the  physical problems could be found in the
papers [34-38].


\section{Superfunctions}

The superfunctions are represented in this package by:
\be
        {\bf bos}(f,0,0),
\ee
for superbosons, while by
\be
        {\bf fer}(g,0,0),
\ee
for superfermions.

The first index denotes the name of the given superobject,  the second
denotes the value of SuSy derivatives,  and the last give the value
of usual derivative. The $bos$ and $fer$ objects are declared as the
operators and as noncom object in the Reduce language. The first index
can take an arbitrary name but with the following restriction:

\be
        {\bf bos}(0,n,m)=0,
\ee
\be
        {\bf fer}(0,n,m)=0.
\ee
for any values of n,m.

        The program has the capability to compute the coordinates of
the arbitrary SuSy expression, using the expansions in the powers of $\theta$.
We have here four commands:
\vspace{0.5cm}

A) In order to have the given expression in the components use
\be
        {\bf fpart}(expression).
\ee
The output is in the form of the list, in which first element is the zero
order term in $\theta$, second is the first order term in $\theta_{1}$,
third is the first order term in $\theta_{2}$ and the fourth is in
$\theta_{2}*\theta_{1}$. For example, the superfunction (11) has the
representation
\pe
  {\bf fpart(bos}(f,0,0)) & => & \{ {\bf fun}(f_{0},0),{\bf gras}(ff_{1},0), \cr
     &&   {\bf gras}(ff_{2},0),{\bf fun}(f_{1},0) \},
\ke
where $fun$ denotes the classical function while the $gras$ the Grassmann
function. First index in the $fun$ or in $gras$ denotes the name of the given
object, while the second denotes the usual derivative.
\vspace{0.2cm}

B) In order to have the bosonic sector only, in which all odd Grassmann
functions disappear,  use
\be
      {\bf bpart}(expression).
\ee

Example:
\be
      bpart(fer(g,0,0)) => \{0, fun(g_{0},0), fun(g_{1},0),0 \}.
\ee

C) In order to have the given coordinates, use
\be
        {\bf bf\underline{~}part}(expression,n),
\ee
where n=0,1,2,3.

Example:
\be
        bf\underline{~}part(bos(f,0,0),3) => fun(f_{1},0).
\ee

D) In order to have the given coordinates in the bosonic sector, use
\be
        {\bf b\underline{~}part}(expression,n),
\ee
where n=0,1,2,3.

Example
\be
        b\underline{~}part(fer(g,0,0),1) => fun(g_{0},0)
\ee


Notice that in the program,  from the default we switch to on the
factor $ fer,bos,gras,fun $. If you remove this factor, then many commands
give you wrong result (for example the command lyst, lyst1 and lyst2).

\section{The inverse and exponentials of superfunctions.}

        In addition to our definitions of the superfunctions we can also
define the inverse and the exponential of superboson.

The inverse of the given bos function (not to be confused with the
`` inverse function '' encountered in the usual analysis) is defined as
\be
        {\bf bos}(f,n,m,-1),
\ee
for an arbitrary $ f,n,m $ with the property $bos(f,n,m,-1)*bos(f,n,m,1)=1$.
The object  $ bos(f,n,m,k) $, in general denotes the k-th power of the
$ bos(f,n,m) $ superfunction.
If we  use the command $``{\bf{let{~}inverse}}''$ then three indices
$ bos $ objects
are transformed onto four indices objects.

The exponential of the superboson function is
\be
        {\bf axp}(bos(f,0,0)).
\ee
It is also possible to use $ axp(f) $, but then we should specify what is f.

We have the following representation in the components for the inverse and
$ axp $ superfunctions
\pe
         fpart(bos(f,0,0,-1)) & = & \{fun(f_{0},0,-1),
       -fun(f_{0},0,-1)*gras(ff_{1},0), \cr
      && -fun(f_{0},0,-1)*gras(ff_{2},0),
      - fun(f_{0},0,-2)*fun(f_{1},0,1)  \cr
      && + 2*fun(f_{0},0,-3)*gras(ff_{1},0)*
       gras(ff_{2},0)\}  \\
      fpart(axp(f)) & = & \{{\bf axx}( bf\underline{~}part(f,0) ),
      axx(bf\underline{~}part(f,0))*bf\underline{~}part(f,1),    \cr
        && axx(bf\underline{~}part(f,0))*bf\underline{~}part(f,2),
        axx(bf\underline{~}part(f,0))  \cr
        && *(bf\underline{~}part(f,3)
        +2bf\underline{~}part(f,1)*bf\underline{~}part(f,2)) \}
\ke

where $ axx(f) $ denotes teh exponentiation of the given classical
function while $ fun(f,m,n) $ the $ n $ th power of the function $ fun(f,m)$.


\section{Ordering.}

Three different superfunctions $ fer,bos,axp $ are ordered among themselves as
\pe
        fer(f,n,m)*bos(h,j,k)*axp(g) , \\
       fer(f,n,m)*bos(h,j,k,l)*axp(g),
\ke

indenpendently of the indices. Superfunctions $ bos $  and $ axp $ are
commuting among themselves, while the superfunctions $fer$ anticommutes
among themselves. For these superfunctions we introduce the following
ordering:.

A) The $ bos $  objects with three and four indices  are  ordered as:
the first index  antilexicographically, the second and the third index as
decreasing order of natural numbers. The last, fourth index is not
ordered because:
\pe
        bos(f,n,m,k)*bos(f,n,m,l) => bos(f,n,m,k+l)
\ke

B) The anticommuting $ fer $ objects we ordered as follows: the first index
antilexicographically, second and third index as decreasing order of natural
numbers.

Example:
\be
fer(f,n,m)*fer(g,k,l)  =>  - fer(g,k,l)*fer(f,n,m)
\ee
for an arbitrary n,m,k,l

\be
fer(f,n,m)*fer(f,n,m) =>     0
\ee
for an arbitrary f,n,m.
\pe
bos(f,2,3,7)*bos(aa,0,3)*bos(f,2,3,-7)   => bos(aa,0,3) , \\
bos(f,2,3,2)*bos(zz,0,3,2)*bos(f,2,3,-2) => bos(zz,0,3,2).
\ke


C) For all exponential functions we have
\be
        axp(f)*axp(g) => axp(f+g).
\ee

\section{(Super)Differential operators.}

        We have implemented three different realizations of the
supersymmetric derivatives. In order to select traditional realization
declare $ {\bf{let {~} trad}} $ . In order to
select chiral or chiral1 algebra declare
$ {\bf{let {~} chiral}} $ or $ {\bf{let {~} chiral1}}$.
By default we have  traditional algebra.

        We have introduced three different types of SuSy
operators which act on the superfunctions
and are considered as operators and as noncomuting objects in
the Reduce language.

For the usual differentiation we introduced two types of operators:

(i) rigth differentations,
\be
{\bf d(1)}*bos(f,0,0) =>  bos(f,0,1)+bos(f,0,0)*d(1);
\ee

(ii) left differentations,
\be
fer(f,0,0)*{\bf d(2)} => -fer(f,0,1)+d(2)*fer(f,0,0).
\ee

From this example follows that the third index in the $bos,fer$ object can
take an arbitrary integer value.

Susy derivatives we denote as $der$ and $del$. $Der$ and $del$ represent
the right and left operatopns, respectively, and are one component argument
operations. The action of these objects on the superfunctions depends on
the choice of the supersymmetric algebra.

Explicitely we have for the traditional algebra:

a) Right SuSy derivative

\pe
        {\bf der(1)}*bos(f,0,0) & =>&  fer(f,1,0)+bos(f,0,0)*der(1), \\
        {\bf der(2)}*fer(g,0,0) & =>&  bos(g,2,0)-fer(g,0,0)*der(2), \\
        der(1)*fer(f,2,0) & =>&  bos(f,3,0)-fer(f,2,0)*der(1), \\
        der(2)*bos(f,3,0) & =>& -fer(f,1,1)+bos(f,3,0)*der(2), \\
        der(1)*bos(f,0,0,-1) & =>& -fer(f,1,0)*bos(f,0,0,-2) + \cr
              && bos(f,0,0,-1)*der(1),  \\
   der(2)*axp(bos(f,0,0)) &=>&  fer(f,2,0)*axp(bos(f,0,0))+ \cr
       && axp(bos(f,0,0))*der(2).
\ke

b) Left SuSy derivative
\pe
        bos(f,0,0)*{\bf del(1)} &=>& -fer(f,1,0)+del(1)*bos(f,0,0), \\
        fer(g,0,0)*{\bf del(2)} &=>&  bos(g,2,0)-del(2)*fer(g,0,0), \\
        fer(f,2,0)*del(2) &=>&  bos(f,3,0)-del(1)*fer(f,2,0), \\
        bos(f,3,0)*del(2) &=>&  fer(f,1,1)+del(2)*bos(f,3,0), \\
        bos(f,0,0,-1)*del(1) &=>&  fer(f,1,0)*bos(f,0,0,-2)+\cr
                && del(1)*bos(f,0,0,-1),\\
   axp(bos(f,0,0))*del(2)& =>& -fer(f,2,0)*axp(bos(f,0,0))+\cr
       && del(2)*axp(bos(f,0,0)).
\ke
From these examples follows that the second index in the fer, bos objects
can take  0, 1, 2, 3 values only with the following meaning: 0 - no SuSy
derivatives, 1 - first SuSy derivative, 2 - second SuSy derivative, 3 - first
and second SuSy derivative.

Using the notations we obtain
\pe
der(1)*der(2)*bos(f,0,0) & => & bos(f,3,0)+ \cr
            && bos(f,0,0)*der(1)*der(2)+ \cr
            &&  fer(f,1,0)*der(2) \cr
            && - fer(f,2,0)*der(1).
\ke

For the ``chiral '' representation, the meaning of the second argument in
the  $bos$ or $fer$ object is  same as in the ``traditional ''
case while the actions of susy operators on the superfunctions are
different. For example we have
\pe
        der(1)*fer(f,1,0) => -fer(f,1,0)*der(1), \\
        der(1)*fer(f,2,0) => bos(g,3,0) - fer(f,2,0)*der(1), \\
        der(2)*bos(g,3,0) => -fer(g,2,1) + bos(g,3,0)*der(2) \\
        bos(g,2,0)*del(2) => del(2)*bos(g,2,)).
\ke
For the ``chiral1'' representation we have different meanig of the second
argument in the $bos$ and $fer$ object. Explicitely  the values 0,1,2 in
this second arguments denotes the values of the susy derivatives while 3
denotes the value of the commutator. Explicitey we have
\pe
der(3)*bos(f,0,0) & => & bos(f,3,0) + 2*fer(f,1,0,0)*der(2) \cr
&& -2*fer(f,2,0)*der(1) + bos(f,0,0)*der(3) \\
        der(1)*fer(f,2,0) &=>& (bos(f,3,0)-bos(f,0,1))/2 - fer(f,2,0)*der(1).
\ke

The supersymmetric operators are always ordered in the case of ``traditional''
algebra as
\pe
        der(2)*der(1) &=>& -der(1)*der(2),\\
        del(2)*del(1) &=>& -del(1)*del(2), \\
        der(1)*del(1) &=>& d(1), \\
        der(1)*del(2) &=>& -del(2)*der(1),
\ke
and similarly for others.

For the ``chiral'' algebra we postulate
\pe
        der(2)*der(1) &=>& -d(1) - der(1)*der(2),\\
        del(2)*del(1) &=>& -d(1) - del(1)*del(2), \\
        der(1)*del(1) &=>& 0, \\
        der(1)*del(2) &=>& -d(1) - del(2)*der(1),
\ke
while for ``chiral1'' additionaly we have
\pe
        der(3)*der(1) => -der(1)*d(1) \\
        der(1)*der(3) => der(1)*d(1) \\
        der(3)*der(2) => der(2)*d(1) \\
        der(2)*der(3) => -der(2)*d(1).
\ke

Please notice that if we would like to have the commponents of some
$bos(f,3,0,-1)$ superfunction in the ``chiral'' representation then new
object appear. Indeed,
\pe
 b\underline{~}part(bos(f,3,0,-1) ,1)  => {\bf fun(f1,0,f0,1,-1)},
\ke
We should consider this five indices object $fun$  as
\pe
fun(f,n,g,m,-k) => (fun(f,n)-fun(g,m)/2)^{-k}.
\ke
Similar interpretation is valid for other commands containing
objects like $bos(f,3,n,-k)$

\section{Action of the operators.}

In order to have the value of the action of the given operator
on some  superfunction we introduce two operations pr and pg.

A)
\be
        pr(n,expression)
\ee
where n:=0,1,2,3.
This command denotes the value itself of action of the SuSy derivatives
on the given expression.For n=0 there is no SuSy derivative, n=1 corresponds
to $der(1)$, n=2 to $der(2)$, while n=3 to $der(1)*der(2)$.

Example:
\be
        pr(1,bos(f,0,0)) => fer(f,1,0),
\ee
\be
        pr(3,fer(g,0,0)) => fer(f,3,0).
\ee


B) For the usual derivative we reserve command
\be
        pg(n,expression)
\ee

where n=0,1,2,...., denotes the value of the usual derivative on the
expression

Example
\be
        pg(2,bos(f,0,0)) => bos(f,0,2)
\ee
\section{Supersymmetric integration}

There is one command ${\bf s\underline {~} int}(number,expression,list)$ 
only. This allows us to compute the value  of supersymmetric integration of 
arbitrary polynomial expression constructed from $fer$ and $bos$ objects. It 
is valid in the traditional representation of the supersymmetry. The 
$numbers$ takes the following values: $ 0 \rightarrow $ corresponds for 
usual $"x"$ integration, $ 1 $ or $ 2 $ for the first or second 
supersymmetric index while $ 3 $ to the integration both over first and 
second indexes. The $list$  is the list of the names of the superfunctions 
over which we would like to integrate. The output of this command is in the 
form of the integrated part and non-integrated part. The non-integrated part 
is denoted by $del(-number)$ if $number = 1,2,3$ and by $d(-3)$ for 0. 

Example 

\be 
{\bf {s\underline {~} int}}(0, 2bos(f,0,1)*bos(f,0,1),\{f\}) = 
bos(f,0,0)^{2},
\ee
\be
s\underline {~} int(1,2*fer(f,1,0)*bos(f,0,0),\{f\}) = bos(f,0,0)^{2},
\ee
\pe
&& s\underline {~} int(3, 
bos(f,3,0)*bos(g,0,0)+bos(f,0,0)*bos(g,3,0),\{f,g\}) =\ \\
&& {~~}{~~}{~~~~}{~~~}{~~}bos(f,0,0)*bos(g,0,0)-\ \\ 
&& del(-3)\Big ( fer(f,1,0)*fer(g,2,0)-fer(f,2,0)*bos(g,1,0) \Big ).
\ke

\section{Integration operators.}

We introduced four different types of integration operators:
right and left denoted as $ d(-1) $ and $ d(-2) $ respectively and moreover
two different types of neutral integration operators $ d(-3) $ and $ d(-4) $.
In first two cases they act acorrding to the formula
\be
        {\bf d(-1)}*bos(f,0,0) = \sum_{i=1}^{\infty} (-1)^{i}*bos(f,0,i-1)*d(-1)^{i},
\ee   \label{calka}

for the right integration, while
\be
        bos(f,0,0)*{\bf d(-2)}= \sum_{i=1}^{\infty} d(-2)^{i}*bos(f,0,i-1),
\ee
for the left integration.


Before using these operators the precision of the integration must be
specified by the declaration
${\bf{ww:=number}}$.
If required this precision can be changed
by clearing the old value of $ww$ and introducing the new one.
Both operators are defined by their action and by the properties
\pe
       d(1)*d(-1) &=& d(-1)*d(1)=d(2)*d(-1)=d(2)*d(-1)=1 ,  \\
             &&   der(1)*d(-1)=d(-1)*der(1),  \\
             &&   d(-1)*del(1)=del(1)*d(-1) ,
\ke
and analogously for $ d(-2) $ and $ der(2), del(2) $.

The neutral operator does not show up any action on some expression but
has several properties. More precisly
\pe
        d(1)*{\bf d(-3)} &=& d(-3)*d(1)=d(2)*d(-3)=d(-3)*d(2)=1,  \\
            &&    der(k)*d(-3)=d(-3)*der(k),  \\
            &&    d(-3)*del(k)=del(k)*d(-3),
\ke
while for $ d(-4) $
\pe
        d(1)*{\bf d(-4)} &=& d(-4)*d(1)=d(2)*d(-4)=d(-4)*d(2)=1 , \\
             &&   der(k)*d(-4)=d(-4)*der(k),
\ke
where k=1, 2.

From the last two formulas we see that $ d(-3) $ operator is transparent under
$ del $  operators while $ d(-4) $ operators stops $ del $  action.

Similarly to  $ d(-3) $ or $ d(-4) $ it is also possible to use the neutral
differentation operator denote ${\bf d(3)}$. It has the properties

\pe
        d(3)*d(-4) &=& d(-4)*d(3)=d(3)*d(-3)=d(-3)*d(3)=1, \\
      &&  der(k)*d(3)=d(3)*der(k), \\
      &&  d(3)*del(k)=del(k)*d(3),
\ke
where k=1, 2.

        We can have also `` accelerated '' integration operators denoted by
$ dr(-n) $ where n is a natural number. The action of these operators
is exactly the same as $ d(-1)**n $ but instead of using n - times the
integration formulas  in the case $ d(-1)**n $, $ dr(-n) $  uses
only once the following formula
\be
{\bf dr(-n)}*bos(f,0,0) = \sum\limits^{ww}_{s=0}(-1)^{s}\pmatrix{ n+s-1 \cr n-1 }
                bos(f,0,s)dr(-n-s).
\ee
We have to, similarly to the $ d(-1) $ case,
declare also the "precision" of integration if we would like to use the
"accelerated" integration operators.
The switch  $ {\bf{let {~} cutoff}} $ and command $ {\bf{cut:= number}} $
allows us to annihilate the higher order terms in the $ dr $ integrations
procedure. Moreover, the switch $ {\bf{let {~} drr}} $
automatically changes usual integrations $ d(-1) $ into
"accelerated" integrations $ dr $. The switch $ {\bf{let {~} nodrr}} $
changes $ dr $ integrations onto $ d(-1) $.


\section{Useful Commands.}

A) Combinations.

We encounter, in many practical applications, problem of construction
of different possible combinations of superfunction and
super-pseudo-differential elements with the given conformal dimensions.
We declare three different procedures in order to realize this requirement:
\pe
     {\bf w\underline{~}comb}(list,n,m,x), \\
     {\bf fcomb}(list,n,m,x), \\
     {\bf pse\underline{~}ele}(n,list,m).
\ke
All these
commands are based on the gradations trick (to associate with
superfuction and superderivative the scaling parametr -
conformal dimension).
We consider here k/2 and k (k natural number and $ k > 0 $ )
gradation only.

        Command $w\underline{~}comb$ gives the most general form of
superfunctions combinations of given gradation. It is four argument
procedure in which:

(i) first argument is a list in which each element is
three elements list in which:  first element is the name of the
superfuction from which we would like to construct our combinations,
second denotes its gradation while the last can take two values
f - in the case where superfunction is superfermionic or b -
for superbosonic.

(ii) second argument is a number - the desired gradation.

(iii) third argument  is an arbitrary not numerical value which enumerates
the free parameters in our combinations.

(iv) fourth argument takes two values

f - in the case when whole combinations should be fermionic  or

b -  for the bosonic nature of combination.
\vspace{0.5cm}

Examples:
\pe
        w\underline{~}comb(\{ \{ f,1,b \},\{g,1,b \} \},2,z,b) & =>&
z1*bos(f,3,0)+
       z2*bos(f,0,1)+\cr
       &&z3*bos(f,0,0)^2; \\
 w\underline{~}comb( \{ \{ f,1,b \} \},3/2,g,f)     &=>&
g1*fer(f,1,0)+ g2*fer(f,2,0);
\ke

        Command $fcomb$, simillarly to  $w\underline{~}comb$, gives us
general form of an arbitrary combination of superfunctions modulo
divergence terms.
It is four argument command with the same meaning of
arguments as in  $w\underline{~}comb$ case. This command first calls
$w\underline{~}comb$, then eliminates in the canonical way
SuSy - derivatives, by integrations by parts of $w\underline{~}comb$.
By canonical we understand that (SuSy) derivatives are removed first
from the superfunction which is first in the list of superfuctions
in fcomb command,  next from  second etc.

In order to illustrate cannonical manner of elimination
of (SuSy) derivatives let us consider some expression which is
constructed from  f, g and h superfunctions and their (SuSy) derivatives.
This expression is first splited onto three subexpression called
$f-expression, g-expression $ and $h-expression$.
$F-expression$ contains  only  combinations of f with
f or g or (and) h, while $g-expression$  contains only  combinations
of g with g or h and last $h-expresion$ contains  only  combinations of
h with h. Command $fcomb$ removes first (SuSy) derivatives from f in f-exprssion,
next from g in g-expression, and finally from h in h-expression.
Let us present such situation on the following example
\be
                fer(f,1,0)*fer(g,2,0) +bos(g,0,0)*bos(g,3,0).
\ee
Let us now assume that we have $ f,g $ order then $ f-expression $ is
$ fer(f,1,0)*fer(g,2,0) $, while $g-expression$ is $ bos(g,0,1)*bos(g,3,0) $.
Now canonical elimination gives us
\be
       - bos(f,0,0)*bos(g,3,0) + 2*bos(g,0,0)*bos(g,3,1),
\ee
while assuming $ g,f $ order we obtain
\be
       - bos(f,3,0)*bos(g,0,0) +2*bos(g,0,0)*bos(g,3,1)
\ee
Example
\pe
 fcomb( \{\{u,1\}\},4,h) &=>& h(1)*fer(u,2,0)*fer(u,1,0)*bos(u,0,0) +\cr
           && h(2)*bos(u,3,0)*bos(u,0,0)^2 + \cr
            && h(3)*bos(u,0,2)*bos(u,0,0)  +\cr
                         &&   h(4)*bos(u,0,0)^4;
\ke

Finally, comand $pse\underline{~}ele$ gives us the general form of
element which belongs to algebra of pseudo-SuSy derivative
algebra [3].
Such element can be symbolically written down as
\be
         ( bos + fer*der(1)+fer*der(2)+bos*der(1)*der(2))*d(1)^n,
\ee
for the traditional and ``chiral'' representation while for ``chiral1''
as
\be
        ( bos + fer*der(1)+fer*der(2)+bos*der(3))*d(1)^n,
\ee
where at the moment, $ bos $ and $ fer $ denotes some  an arbitrary
superfunctions.
The mentioned command  allows us to obtain such element
of the given gradation
which is constructed from some set of superfunctions of given
gradation.  This command is three arguments.
\be
        {\bf pse\underline{~}ele}(wx,wy,wz),
\ee
First index denotes the gradation of SuSy-pseudo-element.
Second the names and gradations of the superfunctions from which we would
like to construct  our element. This second index $ wy $ is in the form of list
exactly the same as in the $ w\underline{~}comb $ command.
Last index denotes
the names which enumerates the free parameters in our combination.
\vspace{0.9cm}

B) Parts of the pseudo-SuSy-differential elements.

In order to obtain the components of the (pseudo)-SuSy element we have
three different commands:
\pe
       {\bf s\underline{~}part}(expression,n),  \\
       {\bf d\underline{~}part}(expression,m),  \\
       {\bf sd\underline{~}part}(expression,n,m),
\ke
where  n,m=0,1,2,3,....

The $s\underline{~}part$ gives us  coefficient standing in n-th SuSy
derivative. However notice, that for n=3 we should consider the coefficients
standing in the $der(1)*der(2) $ operator for the traditional or chiral
representations while for the chiral1 representation the terms standing in
the $der(3)$ operator. The
$d\underline{~}part$ command give us the coefficients
standing in  same power of d(1), while $sd\underline{~}part$ the term
standing in  n-th SuSy derivative and m-th power of usual derivative.

Example:
\pe
ala: &=& bos(g,0,0)+fer(f,3,0)*der(1)+ (fer(h,2,0)*der(2)+\cr
&& bos(r,0,0)*der(1)*der(2))*d(1);\\
s\underline{~}part(ala,3) & => & fer(f,3,0);\\
        d\underline{~}part(ala,1) &=>& fer(h,2,0)*der(2)+\cr
                && bos(r,0,0)*der(1)*der(2);\\
        sd\underline{~}part(ala,0,0) &=>& bos(g,0,0);
\ke
\vspace{0.9cm}

C) Adjoint.

        The adjoint of some SuSy operator is defined in standard form as
\be
        << \alpha,PP*\beta >> = << \beta,PP^*\alpha >>
\ee
where $\alpha$ and $\beta$ are the test superboson functions, PP is the opertor
under consideration and $<< \alpha,\beta >>$ is a scalar product defined as
\be
        << \alpha, \beta >>= \int \alpha*\beta*d\theta_{1}*d\theta_{2}
\ee

where we use the Berezin integral definition [1]
\pe
        \int \theta_{i}*d\theta_{j} = \delta_{i,j}, \\
        \int d\theta_{i} =0.
\ke
For this operation we have  command
\be
        {\bf cp}(expression);
\ee
Examples:
\pe
        cp(der(1))  &=>& -der(1),\\
        cp(del(1)*fer(r,1,0)*der(1))   & =>& fer(r,1,1)+fer(r,1,0)*d(1) -\cr
                                             &&  del(1)*bos(r,0,1),
\ke

From the last example there  follows that it is possible to
define $ cp(del(1)*fer(r,1,0)*der(1))$ in the different but equivalent
manner
namely as $fer(r,1,0)*d(1) - bos(r,0,1)*der(1)$.

From the practical point of view,  we do not define the conjugation for
the $d(-1)$ and $d(-2)$ operators, because then
we should define the precision of the action of the operators $d(-1)$ or
$d(-2)$ and even then, we would obtain very complicated formulas. However,
if somebody decides to use this conjugation to the $d(-1)$ or to the $d(-2)$,
it is recommended, first to change by hand, these operators on $d(-3)$, next
to compute $cp$ and change once more $d(-3)$ into $d(-1)$ or $d(-2)$ together
with the declaration of the precision.
\vspace{0.9cm}

D) Projection.

        In many cases, especially in SuSy approach to soliton theory
we have to obtain projection onto the java.invariant subspace (with respect
to commutator) of algebra of  pseu\-do-Su\-Sy-di\-ffe\-rential algebra.
There are three different subspaces [4] and hence we have two argument
command
\be
        {\bf rzut}(expression,n)
\ee
in which n=0, 1, 2.

Example
\pe
ewa: &=& (bos(f,0,0)+fer(f1,1,0)*der(1)+fer(f2,2,0)*der(2)+\cr
             &&  bos(f3,0,0)*der(1)*der(2))+ (bos(g,0,0)+ \cr
             && fer(g1,1,0)*der(1)+fer(g2,2,0)*der(2)+ \cr
               && bos(g3,0,0)*der(1)*der(2))*d(1),\\
        rzut(ewa,0) & =>& ewa,\\
        rzut(ewa,1) & =>& ewa-bos(f,0,0);\\
        rzut(ewa,2) & =>& bos(f3,0,0)*der(1)*der(2)+
                        (fer(g1,1,0)*der(1) \cr
           && +fer(g2,2,0)*der(2)+ \cr
           &&  bos(g3,0,0)*der(1)*der(2))*d(1),
\ke
\vspace{0.9cm}

E) Analogon of coeff.

Motivated by practical applications, we constructed for our supersymmetric
functions three commands, which allow us to obtain the list of the same
combinations of some superfunctions and (SuSy) derivatives from  some
given operator-valued expression.

    The first command is one argument
\be
        {\bf lyst}(expression)
\ee
with the output in the form of list.

Example
\pe
        magda:=fer(f,1,0)*fer(f,2,0)*a1 + der(1),\\
        lyst(magda) => \{fer(f,1,0)*fer(f,2,0)*a1, der(1) \},
\ke

    The second command is also one argument
\be
        {\bf lyst1}(expression)
\ee
with the output in the form of list in which each element is constructed
from  coefficients and (SuSy) operators of corresponding element in
$lyst$ list. For example
\be
        lyst1(magda) => \{ a1,der(1) \},
\ee

        The third command is also one argument
\be
        {\bf lyst2}(expression)
\ee
with the output in the form of list in which each element is constructed
from coefficients standing in the given expression. For exampla
\be
        lyst2(magda) => \{a1,1\}
\ee
\vspace{0.9cm}

F) Simplifications.

        If we encounter during the process of computations such
expression
\be
        fer(f,1,0)*d(-3)*fer(f,2,0)*d(1)
\ee
it is not reduced further. On the other side we can replace $d(1)$ onto
$d(2)$ and back $d(2)$ onto $d(1)$. In order to do such replacement we
have the command
\be
        {\bf chan}(expression)
\ee
Example
\pe
        && chan(fer(f,1,0)*d(-3)*fer(f,2,0)*d(1)) => \cr
        && -fer(f,2,0)*fer(f,1,0) - fer(f,1,0)*d(-3)*fer(f,2,1).
\ke
Notice that as the result we kill the d(1) operation.
\vspace{0.9cm}

G) O(2) invariance.

In many cases in the supersymmetric theories we deal with the O(2)
invariance of SuSy indices. This invariance follows from the physical
assumption on the nonprivileging the "fermionic" coordinates in the
superspace. In order to check whether our formula posseses such
invariance we can use
\be
        {\bf odwa}(expression)
\ee
This procedure replaces in the given expresion $der(1)$ onto $-der(2)$ and
$der(2)$ onto $der(1)$. Next, it changes, in the same manner, the values
of the action of these operators on the superfunctions.
\vspace{0.9cm}

F) Macierz

Similarly to the representation of the superfunctions in the components
We can define the supercomponent form for the $pse\underline{~}ele$ objects
similarly to the representation of the supersfunctions. Usually we can
consider such object as the matrix which acts on the components of the
superfunctions.It is realized in our program using the command :
\be
        {\bf macierz}(expression,x,y),
\ee
where expression is the formula under consideration while x can take
two values f or b depending wheather we would like to conside bosonic
(b) part or fermionic (f) part of the expression. Last index in this
command denotes the option in which we acts on the bosonic or fermionic
superfunction. It takes two values f- for fermionic test superfunction
or b - for bosonic case. More explicitely we obtain
\pe
macierz(der(1)*der(2),b,f) =\pmatrix{0 & 0 & 0 & 0 \cr
                0 & 0 & d(1) & 0 \cr
                0 & -d(1) & 0 & 0 \cr
                -d(1)**2 & 0 & 0 & 0 } \\
macierz(der(1)*der(2),f,b)= \pmatrix {
                0 & 0 & 0 & 0 \cr
                0 & 0 & 0 & d(1) \cr
                -d(1) & 0 & 0 & 0 \cr
                0 & 0 & 0 & 0 } .
\ke
\section{Functional gradients.}

        In  SuSy soliton approach we very frequently encounter
problem of computing the gradient of the given functional.
The usual definition of the gradient [2] is adopted, in the supersymmetry
also.
\pe
        H^{'}[v] = < grad H ,v > , \\
        H^{'}[v] = \frac{\partial}{\partial \epsilon} H(u+\epsilon v)
        \mid_{\epsilon=0},
\ke
where $ H $ denotes some functional which depends on u. v denotes
vector under which we compute the gradient and $ <,> $ the relevant
scalar product.

We implemented all that in our package for the ``tradicional '' algebra
only. In order to compute the gradient with respect to some superfuction
use
\be
        {\bf gra}(expression,f),
\ee
where "expression" is the given density of the functional, while f denotes
the first index in the superfunction ( name of the superfunction).

Example
\be
        gra(bos(f,3,0)*fer(f,1,0),f) => -2*fer(f,2,1)
\ee
For practical use we perform two additional commands:
\pe
        {\bf dyw}(expression,f) \\
        {\bf war}(expression,f).
\ke
The first computes the variation of expression with respect to
superfunction f, next  removes (via integrations by parts) SuSy-
derivatives from  varied functions and finally produces  list
of factorized $fer$ and $bos$ superfunctions. When the given expression
is full (SuSy)-derivative,  the result of the dyw command is 0 and hence
this command is very usefull in verifications of (SuSy)-divergences of
expressions.

When result of applications of dyw command is not zero
then we would like to have the system of equations on the coefficients
standing in the same factorized $fer$ and $bos$ superfunction. We can quickly
obtain such list using command $war(expression,f)$ with the same
meaning of arguments as in the $dyw$ command.

Examples
\be
        xxx:=fer(f,1,0)*fer(f,2,0)+x*bos(f,3,0)^2;
\ee
\pe
        dyw(xxx,f) &=>& \{ -2*bos(f,3,0)*bos(f,0,0),\cr
        && -2*x*bos(f,0,2)*bos(f,0,0) \}
\ke
\be
        war(xxx,f) => \{-2,-2*x \}.
\ee

\section{Conservation Laws.}
        In many cases we would like to know whether the given expression is
a conservation law for some Hamiltonian equation. We can quikly check it
using
\be
        {\bf dot\underline{~}ham}( {equation},expression)
\ee
where "equation"  is a set of two elements list in which
first element denotes the function while the second its flow.
The second argument should be understand as the density of some
conserved current. For example, for SuSy version of the Nonlinear
Schrodinger Equation  [7] we obtain
\pe
        ew: &=& \{ \{q,-bos(q,0,2)+bos(q,0,0)^3*bos(r,0,0)^2 -\cr
              &&  2*bos(q,0,0)*pr(3,bos(q,0,0)*bos(r,0,0)) \},\cr
              &&\{ r,bos(r,0,2)-bos(q,0,0)^2*bos(r,0,0)^3+\cr
               && 2*bos(r,0,0)*pr(3,bos(q,0,0)*bos(r,0,0)) \} \},\\
        ham: &=& bos(q,0,1)*bos(r,0,0)+x*bos(q,0,0)^2*bos(r,0,0)^2,\\
        yyy: &=& dot\underline{~}ham(ew,ham).
\ke

As the result of previous computations we have a complicated expression
which is not zero. We woulld like to interpreted it as a
full (SuSy)-divergence  and we can quickly verify it, if we use command
$war$. We can solve, obtained list of equations, using known techniques.
For example, in our previous case we obtain
\be
        war(yyy,q) => \{ -4*x,-8*x,-4*x \};
\ee
\be
        war(yyy,r) => \{ 4*x,8*x,4*x \};
\ee
and we conclude that our ham is a constant of motion if x=0.


        It is also possible to use command $dot\underline{~}ham$ to
the pseudo-SuSy-differential element what is very useful in SuSy
approach to Lax operator in which we would like to check
validity of the formula

\be
        \partial_{t}*L:=[ L,A ].
\ee
where $ A $ is a some (SuSy) operator.



\section{Jacobi Identity.}
        The Jacobi identity for some SuSy - hamiltonian operators is verified
using the relation
\be
        << \alpha , P^{`}_{(P\beta)}*\gamma >> + cyclic{~}permutation(
        \alpha,\beta,\gamma),
\ee
where $P^{`}$ denotes the directional derivative along the $P(\beta)$ vector
and $<< , >>$  scalar product. Directional derivative is defined
in the standard manner as [44]
\be
        F^{'}(u)[v] = \frac{\partial}{\partial \epsilon}
        F(u+\epsilon v)\mid_{\epsilon =0},
\ee
where $ F $ is some functional depending on u. V is a directional vector.

In this package we have several commands which allow us to
verify the Jacobi identity.
We have the possibility to compute, indenpendently  of veryfing Jacobi
identity, directional derivative for the given Hamiltonian operator along
the given vector using
\be
                {\bf n\underline{~}gat}( pp, wim )
\ee
where pp is scalar or matrix Hamiltonian operator. $ Wim $  denotes
components of a vector along which we compute derivative and has the
form of list in which each element has following representation
\be
        bos(f) => <expression>.
\ee
The $ bos(f) $, in the last formula, denotes the shift of  $ bos(f,0,0) $
superfunction according to definition of directional derivative.

In order to compute Jacobi identity use command
\be
        {\bf fjacob}( pp, wim),
\ee
with the same meaning of $pp$ and $wim$ as in $n\underline{~}gat$ command.

Notice that ordering of components in  $wim$ list is important and
is connected with interpretation of components of Hamiltonian operator
$pp$ as a set of Poisson brackets constructed just from elements of $ wim $
list.
For example, in our scheme, first component of wim is always connected
with element, from which we create  Poisson bracket and  which
corresponds to first element on the diagonal of pp, second element of
$ wim $ with second element on diagonal of $pp$ and etc.

As the result of applications of $ fjacob $ command to some Hamiltonian
operator we obtain a complicated formula, not necesarily equal to zero but
which would be expressed as (SuSy) divergence. However, we can quickly
verify it using the same method as in $ dot\underline{~}ham $ command
which has been described in previous section.

Usually, after the application of the $ fjacob $  command to some matrix
Hamiltonian operator we obtain the hudge expression which is too complicated
to analyze even when we would like to check its (SuSy)divergence. In this case
we could extract from $fjacob$ expression terms containing given
components of vector test functions fixed by us. We can use in this
order command
\be
                {\bf jacob}(pp,wim,mm)
\ee
where $ pp $ and $ wim $ has the same meaning as in $ fjacob $ case while
$ mm $ is a three elements list denoting the components of
${\alpha,\beta,\gamma}$.

This command is not prepered to compute in full the Jacobi identity,
which contains the integrations operators. We do not implement here the
symbolic integrations of superfunctions in order to simplify the final results.
\newpage

\section{The list of Objects, Commands and Switches}

Objects:
\vspace{0.6cm}

\begin{tabular}{ c  c  c  c  c  c }
& {\bf bos}(f,n,m) & {\bf bos}(f,n,m,k) & {\bf fer}(f,n,m) & {\bf axp}(f)
& {\bf fun}(f,n)  \cr
& {\bf fun}(f,n,m)   & {\bf gras}(f,n)  & {\bf axx}(f)  & {\bf d}(1)
& {\bf d}(2)  \cr
& {\bf d}(3)         & {\bf d}(-1)      & {\bf d}(-2)   & {\bf d}(-3)
& {\bf d}(-4) \cr
& {\bf dr}(-n) & {\bf der}(1)  & {\bf der}(2) & {\bf del}(1) & {\bf del}(2)
\end{tabular}
\vspace{0.3cm}


\noindent Commands
\vspace{0.5cm}

\flushleft
{\footnotesize
\begin{tabular}{ l l l l }
 {\bf fpart}(expression) & {\bf bpart}(expression) &
 {\bf bf\underline{~}part}(expression,n) \cr
 {\bf b\underline{~}part}(expression,n) & {\bf pr}(n,expression) &
 {\bf pg}(n,expression) \cr
 {\bf w\underline{~}comb}
(\{ \{ f,n,x
\},...\} ,m,z,y) &
{\bf fcomb}
(\{ \{ f,n,x
\},...\},m,z,y) &
{\bf pse\underline{~}ele}
(n,\{ \{ f,n \},... \},z) \cr
 {\bf s\underline{~}part}(expression,n) &
{\bf d\underline{~}part}(expression,n) & {\bf sd\underline{~}}(expression,n,m) \cr
 {\bf cp}(expression) & {\bf rzut}(expression,n) & {\bf lyst}(expression) \cr
 {\bf lyst1}(expression) & {\bf lyst2}(expression) & {\bf
chan}(expression) \cr
 {\bf odwa}(expression) & {\bf gra}(expression,f) & {\bf
dyw}(expression,f) \cr
 {\bf war}(expression,f) & {\bf dot\underline{~}ham}(equations,expression)&
{\bf n\underline{~}gat}(operator,list) \cr
{\bf fjacob}(operator,list) & {\bf jacob}(operator,list,\{
$\alpha,\beta,\gamma$ \})& {\bf macierz}(expression,x,y) \cr  
{\bf s\underline {~} int}( numbers, expession,list) & &
\end{tabular}
}
\vspace{0.3cm}


\noindent Switches
\vspace{0.3cm}

\begin{tabular}{ c c c c c c c}
& \bf trad & \bf chiral & \bf chiral1   {~}\bf inverse & \bf drr & \bf nodrr
\end{tabular}

\section{Acknowledgement}
        The author would like to thank to dr. W.Neun for valuable remarks.


\begin{thebibliography}{99}

\item{} J.Wess and J.Bagger, ``Supersymmetry and Supergravity''
        (Princeton, NJ 1982 );

\item{} S.Ferrara and J.G.Taylor ``Introduction to Supergravity''
        ( Moscow 1985 ).

\item{}  L.Faddeev and L.Takhtajan ``Hamiltonian Methods in the Theory of
        Solitons '' (Springer-VerlaG 1987); A.Das ``Integrable Models'' (World
        Sci.1989); M.Ablowitz and H.Segur ``Solitons and the Inverse
        Scattering Transform'' (SIAM Philadelphia 1981).

\item{}  A.Polyakov in ``Fields, Strings and critical Phenomena'' ed.E.Brezin and
        J.Zinn-Justin, (North Holland 1989).


\item{}   S.Manakov, S.Novikov, L.Pitaevski and V.Zakharov ``Soliton Theory: The
        Inverse Problem'' (Nauka, Moscow (1980).

\item{}  L.A.Dickey ``Soliton Equations and Hamiltonian Systems'', (World
        Scientific, Singapore 1991).

\item{}  E.Date, M.Jimbo, M.Kashiwara and T.Miwa, in ``Nonlinear Integrable
        Systems - Classical and Quantum Theory'', ed. by M.Jimbo and
        T.Miwa , (World Scientific, Singapore 1983) p. 39.


\item{} B.Kupershmidt, ``Elements of Superintegrable Systems''(Kluwer 1987).

\item{} M.Chaichian, P.Kulish Phys.Lett.18B (1978) 413.

\item{} R.D'Auria and S.Sciuto, Nucl. Phys. B.171 (1980) 189.

\item{} M.Gurses and O.Oguz, Phys.Lett.108A (1985) 437.

\item{} Y.Manin and R.Radul, Commun.Math.Phys. 98 (1985) 65 .


\item{} C.Morosi and L.Pizzochero, Commun.Math.Phys. 158 (1993) 267 .

\item{} C.Morosi and L.Pizzochero, J.Math.Phys.35 (1994) 2397.

\item{} C.Morosi and L.Pizzochero  ''A Fully Supersymmetric AKNS Theory ''
        preprint of Dipartimento di Matematica,Politecnico di Milano.
        April 1994, to appeare in Commun.math.Phys.(1995).

\item{} P.Mathieu, J.Math.Phys.29 (1988) 2499.

\item{} C.A.Laberge and P.Mathieu, Phys.Lett.215B (1988) 718 .

\item{} P.Labelle and P.Mathieu, J.Math.Phys.32 (1991) 923 .

\item{} M.Chaichian and J.Lukierski, Phys.Lett. 183B (1987) 169 .

\item{} T.Inami and H.Kanno, Commun.Math.Phys. 136 (1991) 519 .

\item{} S.K.Nam, Intern.J.Mod.Phs.4 (1989) 4083.

\item{} K.Hiutu and D.Nemeschansky, Mod.Phys.Lett.A. 6 (1991) 3179.

\item{} C.M.Yung, Phys.Lett. 309B (1993) 75.

\item{} E.Ivanov and S.Krivons, Phys.Lett.291B (1992) 63.

\item{} P.Kulish Lett.Math.Phys.10 (1985) 87.

\item{} G.H.M.Roelofs and P.H.M. Kersten Journ. Math.Phys.33 (1992) 2185 .

\item{} J.C.Brunelli and A.Das, Journ.Math.Phys.36 (1995) 268.

\item{} F.Toppan, Int.Journ.Mod.Phys.A10 (1995) 895.

\item{} S.Krivonos and A.Sorin ``The minimal N=2 superextension of the NLS
        equation'' hep-th/9504084 to appeare in Phys.lett.B.

\item{} S.Krivonos, A.Sorin and F.Toppan `` On the Super-NLS Equation and its
        Relation with N=2 Super-KdV within Coset Approach'' hep-th/9504138.

\item{} W.Oevel and Z.Popowicz, Commun.Math.Phys. 139 (1991) 441.

\item{} Z.Popowicz J.Phys.A:Math.Gen. 19 (1986) 1495.

\item{} Z.Popowicz J.Phys.A:Math.Gen. 23 (1990) 1127.

\item{} Z.Popowicz Phys.Lett.319B (1993) 478.

\item{} Z.Popowicz Phys.Lett. 174A (1993) 411.

\item{} Z.Popowicz Int.Jour.Mod.Phys. 9 (1994) 2001.

\item{} Z.Popowicz Phys.Lett. 194A (1994) 375.

\item{} Z.Popowicz J.Phys.A.Math.Gen. 29 (1996) 1281.

\item{} M.Olshanetsky, Comm.Math.Phys. 88 (1983) 63.

\item{} J.Evans and T.Hollowood, Nucl.Phys. B352 (1991) 723.

\item{} A.C.Hearn ``R E D U C E user's manual 3.6'' (Rand.Publ.1995).

\item{} P.K.H.Gragert and P.H.M.Kersten  ``Liesuper''  ftp.math.utwente.nl/pub/rweb /appl/lisuper.web

\item{} S.Krivonos and K.Thielmans ``A Mathematica Package for Computing N=2
    Superfield Opeerator Product Expansion'', Preprint Imperial College
    London TP  95-96/13 or hep-th/9512029.

\item{} B.Fuchssteiner and A.S.Fokas, Physica 4D (1981) 718.

\end{thebibliography}

\end{document}

